\documentclass[openany, a4paper, 12pt]{report}
\usepackage[utf8]{inputenc}
 
\begin{document}
	
	\begin{titlepage}
		\centering
		\vfill
		{
			\bfseries
			\vskip2cm
			\Large Università di Padova\\
			\vfill
			\Huge Touhou website\\
			\Large Progetto per il corso di recnologie web\\
			\vfill
			\large Bisello Samuele, Cailotto Mirco, Todescato Matteo\\
			\vfill
			Sito web hostato all'indirizzo: www.TODO.com\\ %TODO inserire il sito
			Credenziali per l'amministrazione:\\username: admin, password: admin\\
			Indirizzo email del referente: mirco.cailotto.1@studenti.unipd.it\\
			\vfill
		}    
	\end{titlepage}
	\pagenumbering{roman}
	\tableofcontents
	\newpage
	\pagenumbering{arabic}
	
	
	% Il progetto deve essere accompagnato da una relazione che ne illustri le fasi di progettazione, realizzazione e test ed evidenzi chiaramente il ruolo svolto dai singoli componenti del gruppo. Ricordo che il numero “ideale” di componenti per gruppo è di 3-4 persone. In casi particolari (da concordare col responsabile del corso, Prof. Lamberto Ballan) possono essere costituiti gruppi di 2 persone.
	
	% Nella relazione deve essere riportata una analisi iniziale delle caratteristiche degli utenti che il sito si propone di raggiungere. Le pagine web devono essere accessibili indipendentemente dal browser e dalle dimensioni dello schermo del dispositivo degli utenti. Considerazioni riguardanti diversi dispositivi (laddove possibile) verranno valutate positivamente.
	
	% La relazione deve contenere in prima pagina:
	%indirizzo web del sito;
	%eventuali password degli utenti da utilizzare in fase di correzione (una coppia login-password per ogni classe di utenza), in particolare:
	%l'utente amministratore, se presente, deve avere login e password uguali ad admin;
	%l'utente semplice, se presente, deve avere login e password uguali ad user;
	%indirizzo email del referente del gruppo per eventuali comunicazioni;
	%i file PHP devono avere i permessi corretti;
	%il sito deve utilizzare link relativi in modo da poter essere facilmente installato anche su server o cartelle diverse (se l'installazione necessita di operazioni particolari queste devono essere indicate chiaramente in relazione).
	
	\chapter{Abstract}
	Il progetto sviluppato si propone di sviluppare un sito web che fornisca informazioni riguardo al videogioco Touhou.\\
	Un primo obbiettivo è attirare nuovi utenti che richiedano delle informazioni riguardo al media, che probabilmente hanno conosciuto attraverso qualche immagine satirica sui social network come ad esempio Facebook, Reddit e 4chan o filmati ironici su piattaforme di streaming come ad esempio YouTube.\\
	Il secondo obbiettivo è di fornire informazioni più approfondite a coloro che hanno hanno già una conoscenza base del videogioco, fornendo anche notizie di attualità, in quanto il franchise è in costante espansione, e consentendo loro di commentare per poter creare un senso di comunity.\\
	Un ultimo obbiettivo desiderabile è quello di raccogliere abbastanza utenza per permettere la collaborazione con manifestazioni legate agli ambiti giovani per la realizzazione di conferenze ed eventi dedicati al media.\\
	Gli obbiettivi prefissati si basano sull'operato del sito web Distopia Evangelion (http://distopia.altervista.org, realizzato con Wordpress), che trattano notizie relative a una serie animata giapponese (Neon Genesis Evangelion) e collaborano con altri portali web e con gli organizzatori di alcune manifestazioni per la realizzazione di conferenze.\\
	Gli obbiettivi precedentemente riportati sono da considerarsi puramente teorici e finalizzati alla realizzazione di un sito web con tali caratteristiche, in quanto attualmente non c'è l'intenzione di pubblicazione del sito web.
	
	\chapter{Analisi dell'utenza}
	\section{Descrizione della sessione}
	Questa sezione si pone come obbiettivo l'analisi di tutte le tipologie di utenti che potrebbero navigare sul sito web, ponendo particolare attenzione alle informazioni che cercano.
	\section{Analisi dell'utenza}
	\subsection{Utente che cerca informazioni riguardanti il videogioco}
	La prima tipologia di utenti del sito sono coloro che conoscono il videogioco Touhou per fama, ma non lo hanno mai giocato, per consentire loro di avere una idea immediata sulla tipologia di gioco gli viene presentato in home page alcuni screenshot del videogioco.\\
	Per consentire una informazione più dettagliata è stata creata la pagina gameplay, che è usualmente il primo termine che viene ricercato dai videogiocatori quando desiderano comprendere le meccaniche di un determinato titolo.\\
	Questi utenti usualmente usano come dispositivi pc potenti e con risoluzioni elevate, solitamente uguali o superiori 1080p, e una connessione internet veloce.
	\subsection{Utente che cerca informazioni riguardanti un singolo contenuto}
	Questa categoria di utenti è sicuramente la meno omogenea, sono utenti che vengono a conoscenza di alcune sfaccettature del videogioco attraverso immagini o filmati satirici, definibili anche "memi", presenti in rete e desiderano capirne la fonte.\\
	Questi utenti troveranno nella home, appena sotto gli screenshot del gioco, alcune immagini che faranno capire subito, anche senza guardare il menu, che nel sito sono trattati anche questi aspetti.\\
	L'accesso non sarà eseguito unicamente dalla home page del sito, ma probabilmente potrebbero accedere direttamente alla pagina "Popolarità" attraverso una ricerca effettuata su search engine come Google o Bing, per tale motivo sono state inserite nella sidebar alcune notizie relative al videogioco, che potrebbero attirare l'attenzione e allungare la permanenza nel sito.\\
	Questa tipologia di utenza usualmente ha una buona conoscenza dell'inglese ed usa una vasta gamma di dispositivi recenti, come ad esempio smartphone, tablet e notebook.\\
	\subsection{Utente che cerca approfondimenti}
	Gli utenti che ricercano approfondimenti riguardo al franchise usualmente conoscono e giocano già a Touhou, quindi solitamente navigano utilizzando un pc, ma dato che Touhou è un videogioco 2D che non richiede un hardware potente non si può presupporre l'utilizzo di un pc di ultima generazione.\\
	Usualmente questi utenti hanno una buona conoscenza della lingua inglese e una conoscenza basilare del giapponese, ad esempio riescono a leggere i caratteri hiragana e riconoscono alcuni termini.\\
	Questi utenti cercheranno soprattutto degli approfondimenti riguardanti i molteplici personaggi, dettagli su alcuni capitoli della serie oppure delle informazioni recenti riguardanti il franchise.\\
	A questa tipologia utente è mirata l'ultima parte della home del sito, in quanto sicuramente sarebbero interessati a degli eventi riguardanti Touhou e quindi gli si informa di questa intenzione.
	
	\chapter{Progettazione}
	\section{Descrizione della sessione}
	\section{Nome Sezione}
	\subsection{Nome Sottosezione}
	
	\chapter{Realizzazione}
	\section{Descrizione della sessione}
	\section{Nome Sezione}
	\subsection{Nome Sottosezione}
	
	\chapter{Validazione e test}
	\section{Descrizione della sessione}
	\section{Validazione secondo gli standard}
	\subsection{Nome Sottosezione}
	\section{Test su diversi dispositivi}
	\subsection{Nome Sottosezione}
	\section{Test con diverse impostazioni utente}
	\subsection{Nome Sottosezione}
	\section{Test automatici di accessibilità}
	\subsection{Nome Sottosezione}
	
	\chapter{Ambiguità incontrate e scelte intraprese}
		\section{Descrizione della sessione}
			In questa parte del documento sono elencati tutte le decisioni prese per quanto riguarda gli aspetti ambigui presenti nel sito\\
		\section{Utilizzo della lingua straniera}
			\subsection{Furigana}
				A supporto dei termini scritti in kanji, come si è soliti fare nei siti che presentano termini giapponesi mirati ad adolescenti o a stranieri che potrebbero avere una discreta conoscenza del giapponese, sono stati inseriti i furigana, cioè la pronuncia del termine in caratteri hiragana.
				Questo avviene anche in Giappone in quanto i kanji sono molteplici (si calcola circa 18'000), quindi è frequente che i lettori ne trovino di sconosciuti, quindi li si aiuta inserendo a lato la pronuncia nel sillabario utilizzato per i termini giapponesi, cioè l'hiragana.
				A volte, se il termine è di origine straniera, il furigana potrebbe presentarsi con caratteri del sillabario katanaka, ma questo avviene di rado e non è presente nel nostro sito, in quanto preferiamo presentare le parole inglesi direttamente in alfabeto latino. 
			\subsection{Traslitterazioni}
				Le traslitterazioni dalla lingua giapponese sono spesso imprecise e ci sono controversie sulla pronuncia persino nella stessa lingua giapponese, quindi si è scelto, in caso di screen reader, di lasciare leggere le parole traslitterate usando l'alfabeto italiano, che si rivela anche più preciso rispetto all'inglese nella lettura dei termini.
				Alcune parole giapponesi, che si rilevano essere lette più fedelmente dalla pronuncia inglese, sono state marcate come tali.
				Se le parole traslitterate fossero state marcate come lingua giapponese i termini sarebbero stati considerati in alfabeto romaji (romano, cioè l'alfabeto latino), e quindi traslitterati nuovamente in katakana per poi essere letti usando la lingua giapponese, che avrebbe portato a pronunce errate.
			\subsection{Termini errati}
				Nel sito appaiono termini con delle lettere maiuscole poste senza rispettare la grammatica inglese, in quei casi si è trascritto il termine esattamente come appare nella documentazione ufficiale, all'interno del videogioco o nei file dello stesso, alcuni esempi sono "Nights of Night" o "U.N. Owen Was Her", in questo ultimo caso non ci è dato sapere cosa significhi U.N., si pensa sia un riferimento a Una Nancy Owen, ma non essendoci conferme ufficiali è stato preferito ometterlo.\\
			
	\chapter{Suddivisione del lavoro}
	\section{Descrizione della sessione}
	\section{Suddivisione per membro del gruppo}
	\subsection{Bisello Samuele}
	\subsection{Cailotto Mirco}
	\begin{itemize}
		\item Creazione struttura base del sito
		\item Scrittura contenuti per pagina di home page, gameplay, realizzazione, popolarità
		\item Creazione php per le pagine dinamiche, compresa la parte di amministrazione
		\item Scrittura della funzione php per l'upload delle immagini
		\item Scrittura del sistema di errori forniti all'utente
		\item Parsing degli input per evitare SQLInjection
		\item Creazione sistema di caching per la sidebar
		\item Creazione della struttura del database
		\item Controllo dei valori inseriti dagli utenti generici
		\item Creazione stile per le pagine home page, news, articoli, gameplay, realizzazione, popolarità e capitoli
		\item Stesura dello scheletro della relazione
		\item Stesura parte abstract, di analisi dell'utenza e di ambiguità linguistica della relazione
		\item 
	\end{itemize}
	\subsection{Todescato Matteo}

\end{document}
