\documentclass[openany, a4paper, 12pt]{report}
\usepackage[utf8]{inputenc}
 
\begin{document}
	
	\begin{titlepage}
		\centering
		\vfill
		{
			\bfseries
			\vskip2cm
			\Large Università di Padova\\
			\vfill
			\Huge Touhou website\\
			\Large Progetto per il corso di recnologie web\\
			\vfill
			\large Bisello Samuele, Cailotto Mirco, Todescato Matteo\\
			\vfill
		}    
	\end{titlepage}
	\pagenumbering{roman}
	\tableofcontents
	\newpage
	\pagenumbering{arabic}
	
	\chapter{Nome Capitolo}
	\section{Nome Sezione}
	\subsection{Nome Sottosezione}
	
	\chapter{Nome Capitolo}
	\section{Nome Sezione}
	\subsection{Nome Sottosezione}
	
	\chapter{Ambiguità e scelte intraprese}
		\section{Descrizione della sessione}
			In questa parte del documento sono elencati tutte le decisioni prese per quanto riguarda gli aspetti ambigui presenti nel sito\\
		\section{Utilizzo della lingua straniera}
			\subsection{Furigana}
				A supporto dei termini scritti in kanji, come si è soliti fare nei siti che presentano termini giapponesi mirati ad adolescenti o a stranieri che potrebbero avere una discreta conoscenza del giapponese, sono stati inseriti i furigana, cioè la pronuncia del termine in caratteri hiragana.
				Questo avviene anche in Giappone in quanto i kanji sono molteplici (si calcola circa 18'000), quindi è frequente che i lettori ne trovino di sconosciuti, quindi li si aiuta inserendo a lato la pronuncia nel sillabario utilizzato per i termini giapponesi, cioè l'hiragana.
				A volte, se il termine è di origine straniera, il furigana potrebbe presentarsi con caratteri del sillabario katanaka, ma questo avviene di rado e non è presente nel nostro sito, in quanto preferiamo presentare le parole inglesi direttamente in alfabeto latino. 
			\subsection{Traslitterazioni}
				Le traslitterazioni dalla lingua giapponese sono spesso imprecise e ci sono controversie sulla pronuncia persino nella stessa lingua giapponese, quindi si è scelto, in caso di screen reader, di lasciare leggere le parole traslitterate usando l'alfabeto italiano, che si rivela anche più preciso rispetto all'inglese nella lettura dei termini.
				Alcune parole giapponesi, che si rilevano essere lette più fedelmente dalla pronuncia inglese, sono state marcate come tali.
				Se le parole traslitterate fossero state marcate come lingua giapponese i termini sarebbero stati considerati in alfabeto romaji (romano, cioè l'alfabeto latino), e quindi traslitterati nuovamente in katakana per poi essere letti usando la lingua giapponese, che avrebbe portato a pronunce errate.
			\subsection{Termini errati}
				Nel sito appaiono termini con delle lettere maiuscole poste senza rispettare la grammatica inglese, in quei casi si è trascritto il termine esattamente come appare nella documentazione ufficiale, all'interno del videogioco o nei file dello stesso, alcuni esempi sono "Nights of Night" o "U.N. Owen Was Her", in questo ultimo caso non ci è dato sapere cosa significhi U.N., si pensa sia un riferimento a Una Nancy Owen, ma non essendoci conferme ufficiali è stato preferito ometterlo.\\
\end{document}
